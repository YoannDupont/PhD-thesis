\documentclass[PhD-Yoann-Dupont.tex]{subfiles}
\begin{document}

Cette thèse s'inscrit dans le cadre de la reconnaissance des entités nommées (REN), une discipline cruciale du domaine du Traitement Automatique des Langues (TAL) et particulièrement intéressante pour l'extraction de l'information (EI). Plus particulièrement, nous nous intéressons aux phénomènes de structurations qui entourent les entités nommées. La REN est une tâche capitale du TAL, au début de nombreux traitements au niveau sémantique. Elle sert à l'extraction de relations entre entités nommées, ce qui permet la construction d'une base de connaissance \citep{surdeanu2014overview,rahman2017tac}, l'extraction d'évènements \citep{kumaran2004text}, le résumé automatique \citep{nobata2002summarization,spitz2016terms} ou encore à la conception d'agents conversationnels \citep{cahn2017chatbot}.

%
% introduction
%

Nous commençons cette thèse par une introduction au domaine et à la tâche. Nous montrons qu'il n'y a pas de définition précise communémment acceptée de ce qu'est une entité nommée au-delà de leur nature intrinsèquement référentielle. L'une des premières définitions d'entités nommées vient du MUC-6 \citep{grishman1996message}, où elles étaient définies comme \emph{« tous les noms propres et quantités d'intérêt »}. Elle couvrait les personnes, les lieux et les organisations d'une part, et les expressions temporelles (date et heure) et les expressions numériques (montant monétaire et poucentage) d'autre part. Nous appelons cet ensemble de type l'ensemble "classique" des entités nommées. Déjà à cette première définition d'entités nommées sur un corpus, une forme de structuration, bien que limité, était déjà présente. Il est presque impossible de se passer entièrement de toute forme de structuration lorsque nous parlons d'entités nommées. Même si la plupart des corpus définissent la structuration comme des imbrications, une définition d'entités nommées ayant des structures arborées a été donnée avec Quaero \citep{rosset2011entites}. Nous montrons que la notion d'entité nommées est intrinsèquement liée à leur corpus, raison pour laquelle nous les détaillons dans un chapitre à part entière.

\subsection*{Les corpus d'entités nommées}

Dans ce chapitre, nous détaillons ce qu'est un corpus annoté en entités nommées en détaillant les différentes étapes de sa construction. Nous parlons de la définition du schéma d'annotation, qui définit l'étendue de la tâche à travers les types qui seront étudiés, la prise en compte (ou non) de la structuration. Nous détaillons ensuite le processus d'annotation en lui-même en évoquant les outils qu'il est possible d'utiliser, les guides d'annotation et les formats de fichiers courants. Nous parlons ensuite de l'évaluation de la qualité du corpus via des métriques d'accord inter-annotateurs pour finalement parler du partitionnement du corpus pour évaluation des outils de reconnaissance.

Après avoir décrit ce qu'est un corpus, nous présentons un éventail des corpus que nous avons considérés dans le cadre de cette thèse. Cette liste ne se veut pas exhaustive, de nombreux corpus annotés en entités nommées existants mais n'étant pas nécessairement utilisables dans le cadre de cette thèse. La plupart des corpus que nous présentons contiennent des entités nommées structurées, la plupart des entités nommées ayant une forme arborée. Dans la suite, nous traitons le également French Treebak annnoté en entités nommées, qui n'a pas de strcuturation, afin de créer un système état-de-l'art, mais ce dernier n'est pas le corpus ayant le plus d'intérêt dans le cadre de cette thèse.

Nous présentons ensuite un éventail des méthodes utilisées afin de répondre à la tâche de la reconnaissance d'entités nommées.

\subsection*{Reconnaissance des entités nommées}

Il existe principalement deux types de méthodes afin de répondre à cette tâche. La première consiste à écrire des programmes basés sur des règles explicites et des ressources linguistiques, ces derniers modélisant généralement des conditions dans lesquelles une décision peut être prise de façon sûre. Une autre méthode consiste à appliquer des algorithmes auxquels nous allons donner des exemples afin qu'ils infèrent eux-mêmes des éléments de décision afin de pouvoir au mieux reproduire les exemples à leur disposition, cette approche s'appelant l'\emph{apprentissage automatique}.

Dans ce chapitre, nous commençons par détailler les différentes métriques de qualité qu'il est possible d'utiliser pour évaluer les système effectuant la REN. Nous parlons ensuite de différents systèmes à base de règles. Nous continuons ensuite en détaillant deux approches par apprentissage automatique, que nous pouvons définir comme l'ensemble des méthodes faisant qu'une machine va s'améliorer par l'expérience \citep{cornuejols2011apprentissage}. Plus précisément, nous nous concentrons sur des méthodes  par apprentissage automatique dit supervisé, où des exemples de la tâche à accomplir sont donnés en entrée à un algorithme qui inférera alors des règles basées sur des statistiques afin de pouvoir adhérer au mieux aux exemples qui lui ont été donnés. Les deux approches que nous utilisons sont les CRF \citep{Lafferty01} et les réseaux de neurones \citep{mcculloch1943logical,rosenblatt1958perceptron,elman1990finding}.

Nous terminons ensuite sur un comparatif des différentes méthodes sur le French Treebank. Ce comparatif montrera clairement l'intérêt des méthodes par apprentissage automatique, que nous utilisons tout au long de cette thèse.

\subsection*{Structuration "morphologique" des entités nommées}

Les entités nommées disposent souvent d'éléments discriminants permettant leur identification dans un contexte local. Nous pouvons distinguer deux types d'éléments. Les premiers sont des éléments constitutifs de cette entité et ne peuvent pas lui être retirés : par exemple, une molécule comme "NH$_{4}$Cl" contient pour éléments constitutifs les trois atomes "N", "H" et "Cl". Un autre exemple serait, dans le domaine biomédical, "NF-kappaB", dont les différentes sous-chaînes intéressantes sont "NF", "kappa" et "B". Nous appelons par la suite ces éléments des \emph{affixes caractéristiques} d'une entité nommée. Le second type d'éléments sont des éléments contextuels forts, ces éléments peuvent faire partie ou non de l'entité et ne sont pas toujours présents. Ces éléments contextuels forts sont appelés des \emph{tokens déclencheurs}. Nous pouvons citer par exemple les titres de civilité pour les personnes, les différents types d'organisations (organisation, université, fédération, etc...) ou les différents types de sociétés (société anonyme, à responsabilité limitée, etc.). L'identification de ces éléments est particulièrement importante, autant pour des systèmes à base de règles que pour des système à base d'apprentissage automatique. De manière générale, de grands dictionnaires sont constitués à l'aide de ressources déjà existantes ou de connaissances humaines (où un expert du domaine est alors recommandé). Malheureusement, ces ressources ne sont pas toujours disponibles, que cela soit par leur accès restreint ou leur inexistance pure et simple.

Les travaux visant à décrire les contextes dans lesquels une entité apparaît ont tendance à se baser sur la syntaxe \citep{holat2016fouille} ou sur une liste de tokens déclencheurs dépendant des connaissances des personnes développant des modèles \citep{leaman2013ncbi}. Dans ce chapitre, nous proposons une approche permettant d'extraire la "morphologie" des entités nommées afin de fournir des informations aux systèmes par apprentissage sans besoin de connaissance du domaine. Les approches que nous proposons se rapprochent de la fouille de motifs séquentiels \citep{agrawal1995mining,cellier2010fouille}.

Dans un premier temps, nous détaillons l'algorithme que nous avons mis en place pour extraire les affixes fréquents puis comment les sélectionner. Nous comparons ensuite cette approche avec d'autre se basant sur de la segmentation morphologique plus classiques. Nous montrons ensuite comment notre méthode s'étend naturellement pour l'extraction de tokens déclencheurs et comment nous l'avons utilisée dans le cadre de l'extraction d'entités nommées du Français et pour l'extraction d'adresses avant de conclure.

Nos contributions ici sont multiples. Nous avons développé un algorithme de fouille d'affixes fréquents utilisable pour alimenter un système à base de traits. Nous avons montré que cet algorithme était suffisamment générique pour être appliqué autant dans la fouille de sous-chaînes pour la reconnaissance d'entités nommées chimiques que pour extraire des tokens déclencheurs pouvant être utilisés dans la reconnaissance des entités nommées dans des domaines plus classiques. Ces travaux nous ont permis de publier à la conférence EGC 2016. Nous avons comparé nos méthodes à un algorithme connus de segmentation morphologique et avons montré des gains modeste, mais un bon potentiel. Nous avons ensuite appliqué cette méthode afin d'extraire des tokens déclencheurs dans le French Treebank annoté en entités nommées, ces informations nous ayant permis d'obtenir un système état-de-l'art sur ce corpus. Nous avons également essayé notre méthode sur les adresses postales américaines, où nous nous sommes rapproché de l'état-de-l'art sans pour autant de dépasser.

\subsection*{Les entités nommées structurées}

Dans le chapitre précédent, nous avons vu la structuration des entités nommées d'un point de vue morphologique et syntagmatique. Dans ce chapitre, nous étudions cette structuration d'un point de vue syntaxique. Les annotations dans ce type de modèle ont généralement une structure d'imbrications ou arborée. Une telle structure permet d'obtenir des annotations bien plus riches et informatives qu'une structure plate. Malgré une structure plus complexe, la tâche n'est pas plus difficile pour autant, elle devient même plus simple sur certains aspects. Comme nous l'avons vu pour les adresses, il est plus ardu de reconnaître l'intégralité d'une adresse que de reconnaître d'abord ses éléments constitutifs et d'en reconstruire la structure petit à petit. Pour certaines entités nommées de types plus "classique", cette structuration permet la reconnaissance d'entités qu'il serait particulièrement difficile à reconnaître autrement (par exemple, un stade portant le nom d'une personne).

Il est à noter que la structuration des entités nommées est difficilement évitable : nous avons vu précédemment que l'utilisation de lexiques permettait d'améliorer la qualité des systèmes à base d'apprentissage en leur donnant des informations générales. Par exemple, nous savons humainement qu'une personne a généralement un prénom et un nom de famille et créons généralement un lexique pour chacun d'entre eux. Les tokens déclencheurs permettent également d'offrir une bonne généralisation. La présence d'un titre de civilité devant un nom propre est généralement considéré comme un contexte fort pour extraire une personne. Il en va de même pour les types de société (anonyme, à responsabilité limitée, etc...) ou d'organisation (organisation, fédération, union, etc...). Ces éléments sont dits \emph{constitutifs} d'une entité. La structuration des entités nommées peut alors se formuler comme la reconnaissance d'une entité et des différents éléments qui constituent cette dernière. Par exemple, «\ Maurice Arnoux\ » est une personne dont le prénom est «\ Maurice\ » et le nom est «\ Arnoux\ ». Dans le contexte plus large de l'adresse «\ 1 rue Maurice Arnoux\ », il s'agit également d'un nom de rue. Supposons maintenant que des personnes motivées du laboratoire Lattice décident d'organiser le «\ tournoi de tennis 1 rue Maurice Arnoux\ »\footnote{Le laboratoire Lattice étant situé au 1 rue Maurice Arnoux.}, l'adresse fera alors partie d'une entité de type évènement. Nous pouvons voir avec l'exemple précédent que la structure des entités est accumulative, se construisant petit à petit. Cet exemple d'une adresse peut s'étendre à d'autres entités, comme un complexe sportif, ou une société, pour lesquels il n'est pas rare qu'ils portent le nom d'une personne.

La notion d'entité nommée pose certaines questions quant à l'éventail des types intéressants à analyser. Différentes définitions aux étendues variables ont été proposées, pouvant recouvrir de 5 à 10 types \citep{grishman1996message,tjong2003introduction,sagot2012annotation} à environ 200 \citep{sekine2002extended}. Définir précisément ce qu'est une entité nommées et comment ces dernières sont construites pose encore plus de questions lorsque la structuration doit être prise en compte. Par exemple, la (non-)prise en compte du titre ou de la fonction pour les personnes changera son étendue et pourra ajouter le type "titre" à la liste de types reconnus. Les montants illustrent assez bien cette question du périmètre : doit-on annoter uniquement les montants monétaires (euros, dollars, etc...) ? Les montants mesurables selon une unité de mesure (heures, grammes, kilomètres/heure, honoraires) ? Ou l'ensemble des éléments quantifiés (quatre doctorants, huit cafés) ? La structuration des entités soulève également des questions qui lui sont propres. Dans le cadre classique, le nom d'un pays peut être utilsé pour désigner un lieu ou son gouvernement, chacun de ces cas étant annoté différemment. Lorsque les entités sont structurées, nous pouvons nous poser la question de comment annoter le gouvernement d'un pays. Doit-on d'abord annoter le pays avec une annotation de type organisation par dessus, puisque le gouvernement est indissociable de son pays ? Doit-on seulement annoter le gouvernement et pas le lieu, puisque nous parlons effectivement d'un gouvernement ? Bien que ce chapitre ne propose pas de réflexion approfondie sur le sujet de l'annotation structurée en soi, ces problématiques méritent d'être exposées car elles montrent les problèmes spécifiques à l'annotation en entités nommées structurées.

Nos contributions dans ce chapitre sont les suivantes. Premièrement, nous proposons un type de cascade d'étiqueteurs linéaires qui n'avait jusqu'à présent jamais été utilisé pour la reconnaissance d'entités nommées, généralisant les approches précédentes qui ne sont capables de reconnaître des entités de profondeur finie ou ne pouvant modéliser certaines particularités des entités nommées structurées. Nous apportons également comme contribution un comparatifs entre les CRF et les réseaux de neurones, où nous montrons un avantage clair des réseaux de neurones pour la tâche, ces derniers modélisant mieux les dépendances qui peuvent exister d'un niveau à l'autre. Les travaux que nous avons menés pour décrire ce chapitre nous ont permis de publier dans une conférence internationale, à savoir CICling 2017.

L'un des problèmes principaux concernant les corpus avec des annotations stucturées est leur disponibilité. En effet, ces corpus sont des ressources très rares, il n'en existe aucun pour l'anglais à notre connaissance. Dans ce chapitre, nous nous concentrons sur le corpus Quaero, l'un des rares corpus à proposer des annotations en entités nommées structurées.

\subsection*{Ouvertures}

    %
    % SEM
    %


\subsubsection*{Apprentissage actif}

L'\textit{active learning} (apprentissage actif) \citep{angluin1987learning,kinzel1990improving,baum1991neural,mackay1992information}, est un type d'apprentissage semi-supervisé, dans lequel sont utilisés autant des données annotées que non annotées. L'apprentissage actif repose sur le principe qu'un apprenant (ici, un algorithme d'apprentissage automatique) est capable de requêter un oracle afin d'obtenir la sortie véritable sur de nouvelles données non annotées. Dans ce type d'apprentissage, nous avons typiquement un grand volume de données non annotées et très peu voire pas du tout de données annotées. L'\textit{apprentissage actif} est souvent utilisé dans le cas où le coût d'annoter de nouvelles données est important ou que les données annotées sont inexistantes malgré une tâche connue. Cette méthode permet d'accélérer le processus d'annotation ou de réduire le volume de données nécessaire à accomplir une tâche données. L'oracle est généralement un être humain, mais il peut également être une annotation de référence déjà connue dans le cas où l'on cherche à mesurer la vitesse d'annotation ou le volume de données minimum nécessaire. Le principe général de l'\textit{apprentissage actif} se résume dans une boucle dans laquelle l'apprenant va annoter des données inconnues et proposer les exemples les plus pertinents à l'oracle afin d'enrichir le jeu de données annotées.


\subsubsection*{Relations entre entités nommées}

Dans cette perspective, nous irons plus loin que l'extraction des entités nommées, en extrayant leurs relations. L'extraction de relations entre entités nommées est la suite logique de la reconnaissance des entités nommées. Il s'agit ici de structurer une connaissance à l'échelle, non plus des tokens, mais du document. Lorsque l'on extrait des relations entre entités nommées, la structuration que l'on peut extraire des entités peut s'avérer particulièrement utile. Si l'on cherche par exemple à savoir si deux personnes sont apparentées, le fait qu'elle ait le même nom de famille est en général un bon indice. Un autre indice intéressant pour établir la relation entre deux entité est dans la présence de certains tokens. La présence d'un token comme "frère" ou "s\oe ur" entre deux personnes est également à un indice pour décider de s'ils sont apparentées, bien que cela ne soit pas suffisant.

Dans cette perspective, nous donnons une vision des travaux préliminaires que nous avons effectués dans le cadre du projet \textit{Multimedia Multilingual Integration} (IMM) de l'Institut de Recherche Technologique SystemX (IRT SystemX). Le projet étant assez conséquent et notre contribution assez spécifique, nous détaillons donc également des contributions d'autres membres de l'IRT en plus des nôtres afin d'offrir le portrait minimal nécessaire pour la compréhension du chapitre. Dans le cadre de ce projet, nous avons aidé à produire une chaîne de traitements afin de construire automatiquement une \emph{base de connaissance} en participant au module d'extraction de relations entre mentions d'entités nommées. Ces travaux menés dans le cadre du projet IMM m'ont permis de participer au défi TAC-KBP 2016 \citep{rahman2017tac} ainsi que de contribuer à une démo à JEP-TALN 2016 \citep{mesnard2016}.

Dans cette perspective, nous détaillons d'abord la tâche à laquelle nous nous sommes attaquée, à savoir l'extraction de relations dans le cadre du défi TAC-KBP 2016. Nous détaillons ensuite l'approche de \emph{supervision distante} avec laquelle nous avons constitué un corpus d'apprentissage pour répondre à la tâche. Puis, nous détaillons l'outil que nous avons utilisé pour détecter les relations entre mentions d'entités nommées, à savoir MultiR, et de comment nous l'avons configuré pour répondre à la tâche. Nous détaillons ensuite les résultats que nous avons obtenus et concluons.

\end{document}