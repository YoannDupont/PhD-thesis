\documentclass[PhD-Yoann-Dupont.tex]{subfiles}
\begin{document}

Les entités nommées disposent souvent d'éléments discriminants permettant leur identification dans un contexte local. Nous pouvons distinguer deux types d'éléments. Les premiers sont des éléments constitutifs de cette entité et ne peuvent pas lui être retirés : par exemple, une molécule comme "NH$_{4}$Cl" contient pour éléments constitutifs les trois atomes "N", "H" et "Cl". Un autre exemple serait, dans le domaine biomédical, "NF-kappaB", dont les différents sous-chaînes intéressantes sont "NF", "kappa" et "B". Nous appellerons par la suite ces éléments des \emph{affixes caractéristiques} d'une entité nommée. Le second type d'éléments sont des éléments contextuels forts, ces éléments peuvent faire partie ou non de l'entité et ne sont pas toujours présents. Ces éléments contextuels forts seront appelés des \emph{tokens déclencheurs}. Nous pouvons citer par exemple les titres de civilité pour les personnes, les différents types d'organisations (organisation, université, fédération, etc...) ou les différents types de sociétés (société anonyme, à responsabilité limitée, etc.). L'identification de ces éléments est particulièrement importante, autant pour des systèmes à base de règles que pour des système à base d'apprentissage automatique. De manière générale, ces systèmes intègrent de grands dictionnaires constitués à l'aide de ressources déjà existantes ou de connaissances humaines (où un expert du domaine est alors recommandé). Malheureusement, ces ressources ne sont pas toujours disponibles, que cela soit par leur accès restreint ou leur inexistance pure et simple.

Nous appellerons par la suite (et pas abus de langage) ces éléments la \emph{morphologie} d'une entité nommée, qui sera le sujet d'étude de ce chapitre. Ces \emph{tokens} étant des sous-éléments communs à de nombreuses entités, nous avons créé un algorithme, se basant sur la recherche des sous-chaînes communes, afin de constituer des lexiques d'élements constitutifs des entités nommées. et montrerons comment nous l'avons utilisé pour extraire autant les éléments constitutifs que les contextes forts. Nous nous concentrerons sur trois tâches principales : la reconnaissance d'entités nommées chimiques, la reconnaissance d'adresses et nous établirons un système état-de-l'art pour les entités nommées du Français.

Les travaux visant à décrire les contextes dans lesquels une entité apparaît ont tendance à se baser sur la syntaxe \citep{holat2016fouille} ou sur une liste de tokens déclencheurs dépendant des connaissances des personnes développant des modèles \citep{leaman2013ncbi}. Dans ce chapitre, nous proposons une approche permettant d'extraire la "morphologie" des entités nommées afin de fournir des informations aux systèmes par apprentissage sans besoin de connaissance du domaine. Les approches que nous proposons se rapprochent de la fouille de motifs séquentiels \citep{agrawal1995mining,cellier2010fouille}.

Ce chapitre s'articulera de la manière suivante : dans un premier temps, nous détaillerons l'algorithme que nous avons mis en place pour extraire les affixes fréquents puis comment les sélectionner. Nous comparerons ensuite cette approche avec d'autre se basant sur de la segmentation morphologique plus classiques. Nous monterons ensuite comment notre méthode s'étend naturellement pour l'extraction de tokens déclencheurs et comment nous l'avons utilisée dans le cadre de l'extraction d'entités nommées du Français et pour l'extraction d'adresses avant de conclure.

Nos contributions ici sont multiples. Nous avons développé un algorithme de fouille d'affixes fréquents utilisable pour alimenter un système à base de traits. Nous avons montré que cet algorithme était suffisamment générique pour être appliqué autant dans la fouille de sous-chaînes pour la reconnaissance d'entités nommées chimiques que pour extraire des tokens déclencheurs pouvant être utilisés dans la reconnaissance des entités nommées dans des domaines plus classiques. Ces travaux nous ont permis de publier à la conférence EGC 2016. Nous avons comparé nos méthodes à un algorithme connus de segmentation morphologique et avons montré des gains modeste, mais un bon potentiel. Nous avons ensuite appliqué cette méthode afin d'extraire des tokens déclencheurs dans le French Treebank annoté en entités nommées, ces informations nous ayant permis d'obtenir un système état-de-l'art sur ce corpus. Nous avons également essayé notre méthode sur les adresses postales américaines, où nous nous sommes rapproché de l'état-de-l'art sans pour autant de dépasser.

\end{document}