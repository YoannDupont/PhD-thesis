\documentclass[PhD-Yoann-Dupont.tex]{subfiles}
\begin{document}

Nous avons montré dans ce chapitre une méthode simple et adaptable afin d'extraire la \emph{morphologie} des entités nommées. Cette méthode permet de soit récupérer les éléments constitutifs, se rapprochant alors d'une segmentation, soit d'extraire les contextes déclencheurs forts, auquel cas cette méthode est plus une assistance à la création de lexiques. Nous l'avons appliquée avec succès sur la tâche CEM qui utilise le corpus CHEMDNER et avons montré des résultats comparables à ceux obtenus en utilisant Morfessor. Nous avons cependant vu qu'une méthode bien plus rudimentaire consistant à générer l'ensemble des sous-chaînes donnait une meilleure qualité finale, même si d'un même ordre de grandeur. Cela est malheureusement demeuré insuffisant pour prétendre à des résultats état-de-l'art. Beaucoup de travail doit encore être fait de ce côté, nous pensons nous orienter vers des modèles à base de réseaux de neurones, particulièrement des réseaux de neurones intégrant des convolutions au niveau des caractères. Ces réseaux permettraient alors d'obtenir une représentation dense des sous-séquences, qui paraît bien plus adaptée à la tâche qu'une représentation creuse, notamment vu la faiblesse des résultats que nous avons obtenus sur les entités inconnues. Il a été montré notamment que les réseaux de neurones avaient une meilleure capacité de généralisation que les CRF, donnant de meilleurs résultats sur les entités inconnues, constat que nous avons également fait dans la section \ref{sec:taxonomy-ftb}.

Nous avons également constaté tout au long de ce chapitre que l'extration de la morphologie des entités nommées, bien qu'une bonne source de traits pour des systèmes par apprentissage automatique, reste insuffisante par elle-même. La tâche de REN demeure compliquée et il est impossible d'y répondre sans passer par des systèmes plus complexes demandant plusieurs passes de traitement, comme nous l'avons vu dans les sections \ref{sec:ontology-addresses} et \ref{sec:taxonomy-ftb}.

Les systèmes ainsi créés permettent cependant d'envisager d'aller plus loin, nous pouvons ainsi songer reconnaître la structure des entités nommées.

\end{document}