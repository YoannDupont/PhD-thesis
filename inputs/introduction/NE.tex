\documentclass[PhD-Yoann-Dupont.tex]{subfiles}
\begin{document}

Comme vu dans la section \ref{sec:EI-introduction}, l'extraction d'information (EI) est une tâche qui consiste à retrouver de façon automatisée les éléments d'intérêt présents dans des documents, ainsi que de les mettre en relation les uns avec les autres. Dans le cadre de cette thèse, nous nous concenterons sur un type d'information particulier, appelées entités nommées. Plus particulièrement, nous nous concentrerons sur la tâche de leur extraction, appelée la \emph{reconnaissance d'entités nommées} (NER), cette extraction sera faite sur les textes écrits. Il s'agit d'une tâche très importante du TAL qui sert généralement de point de départ à d'autres tâches telles que l'extraction de relations \citep{bunescu2005}, la construction d'une base de connaissance \citep{surdeanu2013overview}, l'entity linking \citep{doddington2004automatic}, la résolution de coréférence \citep{denis2009global,durrett2014joint,hajishirzi2013joint}, le résumé automatique \cite{gupta2011named}, les systèmes de questions-réponses \citep{han2017answer}, etc. Elle permet, plus largement, \emph{l'accès à l'information} \citep{nouvel2015entites} pertinente pour des tâches qui autrement seraient irréalisables.

Il n'y a pas de définition précise communémment acceptée de ce qu'est une entité nommée. Tous s'accordent cependant qu'une entité nommée est une unité linguistique de nature référentielle. Prenons l'exemple des personnes, communément admises comme étant des entités nommées. Si, dans un texte figure «\ le président Français Emmanuel Macron\ », nous savons avec certitude que nous référons à une unique personne clairement identifiable\footnote{cela n'est vrai que parce que l'exemple ici, même sans contexte, n'est pas ambigu}. Une instance dans le texte d'une entité nommée est appelée une \emph{mention}. Passé cet aspect, des divergences fondamentales existent quant à la nature d'une entité nommée. \citet{linguistic2005ace} indique qu'une entité nommée doit avoir la propriété d'\emph{unicité référentielle}, c'est-à-dire qu'une mention ne renvoie qu'à une unique entité nommée. \citet{poibeau2005statut}, quant à lui, conteste l'unicité de cette référence et préfère les qualifier de \emph{dénotationnelles}, ce qui implique une certaine intention à l'inverse de la référence. Il indique notamment le côté hautement polysémique de certaines entités qu'il n'est pas toujours possible de distinguer. C'est notamment le cas des événements souvent mentionnés à l'aide de la date où ils sont survenus.

L'une des premières définitions d'entités nommées vient de la campagne \textit{Message Understanding Conference 6} (MUC-6) \citep{grishman1996message}, où elles étaient définies comme \emph{« tous les noms propres et quantités d'intérêt »}. Elle couvrait les personnes, les lieux et les organisations d'une part, et les expressions temporelles (date et heure) et les expressions numériques (montant monétaire et poucentage) d'autre part. L'idée de cette définition était d'être le plus simple possible, aucun élément constitutif n'avait à être retrouvé. Pour les personnes, la reconnaissance des prénoms et noms de familles n'était nullement important, seule l'identification était utile. Il y a en revanche une exception à cette règle : pour les dates et heures, si une ville était mentionnée pour indiquer le fuseau horaire, la ville devait également être identifiée. Par exemple, «\ 1:30 p.m. Chicago time\ » est une expression temporelle à l'intérieure de laquelle le lieu «\ Chicago\ » soit également être trouvé \footnote{exemple repris de \url{http://www.cs.nyu.edu/cs/faculty/grishman/NEtask20.book_16.html\#HEADING43}}. Nous pouvons déjà voir que, dès leur création, les entités nommées avaient un certain besoin de structuration de l'information afin d'être suffisamment précise.

Plus tard, la campagne \textit{Automatic Content Extraction} (ACE) \citep{doddington2004automatic} a donné comme périmètre au entités nommées les personnes, les organisations, les lieux, les bâtiments, les armes, les véhicules et les entités géo-politiques. Ces entités pouvaient être raffinées à l'aide de sous-types. Les entités nommées étaient structurées selon un schéma d'imbrications, des entités pouvant se recouvrir les unes les autres. Par exemple, l'entité de type personne «\ le président Français Emmanuel Macron\ » contiendrait également l'entité «\ Emmanuel Macron\ » de même type. Cette campagne incluait plusieurs tâches connexes à la reconnaissance d'entités nommées, à savoir la reconnaissance des relations qui les lient, ainsi que l'extraction d'évènements. Ici, le côté applicatif de la reconnaissance d'entités nommées est on ne peut plus clair.

\citet{sekine2004} a quant à lui défini 150 type d'entités nommées organisés de façon hiérarchique. Le but de cette définition étant de donner une définition générale de ce qu'est une entité nommée pour comprendre un maximum de cas d'usage. Si un cas d'usage s'avère plus particulier, il est possible de n'utiliser qu'une partie de la hiérarchie.

\citet{ehrmann2008entites} a proposé la définition d'entités nommées suivante, que nous utiliserons par la suite : \begin{quote}«\ Étant donnés un modèle applicatif et un corpus, on appelle entité nommée toute expression linguistique qui réfère à une entité unique du modèle de manière autonome dans le corpus.\ »\end{quote}

La notion de modèle applicatif de cette définition sert à indiquer que ce que l'on considère comme entité nommée peut changer pour de nombreuses raisons, la plus évidente étant l'application à un domaine différent. Il est assez peu intéressant, par exemple, de reconnaître les personnes dans les textes parlant de chimie, tout comme reconnaître des protéines dans des articles de journal d'information. Le modèle applicatif rend bien compte que la notion d'entités nommées n'est pas autonome, un même corpus pouvant être annoté très différemment selon la finalité du modèle. Cette notion d'entités nommées est souvent rattachée à son domaine d'application lorsque ce dernier diffère totalement de celui défini par \citet{grishman1996message} : on parle par exemple d'\emph{entités nommées biomédicales} ou d'\emph{entités nommées chimiques}. Cette définition montre également l'importance du corpus pour cette tâche, ce qui sera le sujet de notre prochaine section, où nous détaillerons un ensemble de corpus que nous avons étudié et que nous avons retenus (ou non) selon qu'ils correspondaient à notre domaine applicatif (ou non).

Nous avons vu qu'en ce qui concerne les entités nommées, il est presque impossible de se passer entièrement de toute forme de structuration. Même si la plupart des corpus définissent la structuration comme des imbrications, une véritable structuration des entités nommées est donnée par les annotations de type Quaero \citep{rosset2011entites}. Il s'agit de l'un des premiers, si ce n'est le premier, guide d'annotation à proposer une véritable structuration des entités nommées, au-delà de la simple imbrication.

%Depuis leur définition dans le MUC-6, les entités nommées ont eu des classifications de plus en plus raffinées, couvrant plus d'éléments de nature différente et/ou affinant le grain sur les entités déjà définies. Par exemple, \citet{sekine2002extended} propose 150 types d'entités nommées hiérarchisés. Le besoin de structuration des entités nommées est apparu relativement rapidement, même si les premiers corpus en proposant une offraient seulement une structure d'imbrications, où un même ensemble d'entités est utilisé pour annoter des séquences de plus en plus longues. C'est notamment le cas pour les corpus de la tâche 9 de Sem'Eval 2007 \citep{marquezSemEval2007}. L'un des premiers corpus à proposer une véritable structuration des entités nommées est le corpus Quaero \citep{rosset2011entites}, un corpus de Français oral transcrit d'actualités.

Comme nous l'avons vu plus haut, la reconnaissance d'entités nommées est un domaine très large et la notion d'entités nommées est très dépendante du corpus utilisé. C'est pour cette raison que nous allons dans la prochaine section nous concentrer sur les différents corpus que nous avons considérés dans le cadre de cette thèse. Après avoir présenté les différents corpus d'entités nommées, nous présenterons les différentes méthodes qu'il existe à l'heure actuelle pour répondre à cette tâche. Nous étudierons ensuite la "morphologie" d'une entité nommée et comment l'extraire afin d'alimenter des algorithmes par apprentissage. Nous verrons ensuite les entités nommées structurées dont la forme est typiquement arborescente. Nous verrons ensuite l'extraction des relations entre entités nommées avant de conclure.

\end{document}