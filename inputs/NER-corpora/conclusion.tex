\documentclass[PhD-Yoann-Dupont.tex]{subfiles}
\begin{document}

Dans ce chapitre, nous avons fait un tour d'horizon de quelques corpus annotés en entités nommées que nous avons considérés au cours de la thèse. Nous avons montré que la plupart des corpus structurent les entités nommées par imbrications d'entités du même type, comme c'est le cas pour le Genia ou SemEval 2007. Beaucoup de corpus ne proposent pas une structuration des annotations à proprement parler (ou une très restreinte). Ils demeurent intéressants car il proposent soit une annotation fondamentalement structurée comme des adresses ou permettent d'étudier plus particulièrement les régularités syntagmatiques des entités. L'annotation de type Quaero propose une annotation arborée des entités nommées en distinguant deux sous-classes : les composants et le types. Le corpus Quaero est en ce sens assez unique par rapport aux autres corpus présentés ici.

Nous avons vu que les corpus, ressource essentielle pour la reconnaissance d'entités nommées, n'avaient que trop rarement accord inter-annotateur de quelque sorte que ce soit. Il s'agit d'un problème récurrent des corpus dans le domaine du TAL, dont la qualité ne peut pas être attestée de prime abord. Ces manques d'accords inter-annotateurs sont rarement dus à une mauvaise intention de ceux qui produisent les données, ils reflètent plus un manque de moyens généralisé, autant au niveau humain que financier. Parmi les causes principales d'un manque d'un accord inter-annotateurs, nous trouvons : il n'y a qu'un seul annotateur, le manque de temps pour produire un accord inter-annotateurs. En supposant que nous ayons un échantillon représentatif du corpus annoté par deux annotateurs, le calcul d'un $\kappa$ pose certains problèmes. Comme l'indiquent \citet{alex2010agile,grouin2011proposal}, le $\kappa$ suppose de connaître par avance le nombre d'éléments de références, ce qui n'est généralement pas possible dans le cadre des entités nommées. Il n'est donc généralement pas possible d'obtenir un $\kappa$ exact. La meilleure approximation à notre avis est celle donné par \citet{grouin2011proposal} est de considérer l'ensemble des entités qu'au moins un annotateur a annoté. Cette représentation a l'avantage de ne considérer que les proposition humaines, donc l'objet de l'accord.

L'utilisation d'outils spécialisés pour l'annotation des entités nommées semble obligatoire. Ces derniers doivent incorporer une sélection d'une partie du corpus afin qu'un accord inter-annotateurs soit calculé de manière automatique. Idéalement, ces derniers devraient offrir la possibilité de fournir des candidats aux utilisateurs afin d'accélérer le processus de découverte des entités nommées. Ces fournisseurs de candidats doivent être couvrants afin de minimiser le nombre d'entités manquées, et être un minimum précis afin de ne pas submerger les annotateurs de candidats incorrects.

Nous avons présenté dans cette partie divers corpus pour la reconnaissance d'entités nommées. Nous avons montré que cette tâche a de nombreux jours, autant dans les domaines d'application que dans leur définition même. Nous avons vu en quoi la tâche pouvait être plus ou moins complexe et demandeant des méthodes plus ou moins puissantes pour pouvoir être traitées. Les différentes notions d'entités nommées de différents domaines demandent également des traitements particuliers à tâche équivalente : une approche utilisant des dictionnaires pour les cas du FTB ou de Quaero est particulièrement adapté, mais ne saurait être suffisante pour traiter des entités biomédicales ou chimiques de Genia ou CHEMDNER, qui requierent une analyse morphologique beaucoup plus fine. La tâche d'entités nommées peut également être plus complexe dans sa définition que la simple reconnaissance de sous-chaînes, les corpus comme Genia et le Quaero disposant d'une structuration sur plusieurs niveaux, les méthodes capables de traiter les données plus simples étant alors incapables de les traiter.

Dans la partie suivante, nous présenterons différentes méthodes généralement utilisées afin de répondre à ces différentes tâches. Nous présenterons un éventail qui se veut suffisant avant de donner notre choix pour la tâche.

\end{document}