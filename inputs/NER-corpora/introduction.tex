\documentclass[PhD-Yoann-Dupont.tex]{subfiles}
\begin{document}

Comme nous l'avons vu précédemment, les entités nommées sont intrinsèquement liées à leur corpus. Dans ce chapitre, nous parlerons principalement de ces derniers afin de présenter un éventail des corpus que nous avons considérés dans le cadre de cette thèse. Cette liste ne se veut pas exhaustive, de nombreux corpus annotés en entités nommées existants mais n'étant pas nécessairement utilisables dans le cadre de cette thèse. Dans un premier temps, nous parlerons des différentes mesures de qualité afin d'évaluer les annotations produites par les humains. Nous parlerons ensuite plus en détail des différents corpus que nous avons considérés. La plupart des corpus que nous présenterons contiennent des entités nommées structurées, la plupart des entités nommées ayant une forme arborée. Dans la suite, nous traitons le également French Treebak annnoté en entités nommées, qui n'a pas de strcuturation, afin de créer un système état-de-l'art, mais ce dernier n'est pas un corpus ayant le plus d'intérêt dans le cadre de cette thèse.

\end{document}