\documentclass[PhD-Yoann-Dupont.tex]{subfiles}
\begin{document}

Dans ce chapitre, nous avons décrit une méthode générale pour la reconnaissance d'entités nommées structurées par modèles linéaires. Nous avons donné une procédure générique et l'avons adaptée pour mieux correspondre à la structure propre aux entités du corpus Quaero. Nous avons montré que cette approche était justifiée, la qualité obtenue étant compétitive. Bien que nous n'avons pas pu améliorer l'état-de-l'art avec nos approches, la variante \emph{bootstrap} nous a permis d'améliorer les résultats par rapport aux méthodes équivalentes, ce qui nous laisse confiants par rapport aux résultats que nous pouvons obtenir.

Nous avons caractérisé les erreurs les plus communes afin de quantifier les différents manques à gagner de nos systèmes, ce qui nous a donné des pistes pour les améliorer et nous a permis d'identifier des erreurs humaines. Nous pourrions estimer la proportion des erreurs propagées en vérifiant les types et frontières des entités plus basses. Par exemple, si nous avons une erreur de type sur une personne, vérifier les composants permettrait de voir immédiatement si l'erreur vient des niveaux inférieurs. Un autre test que nous pourrions effectuer par la suite serait de comparer les systèmes appris ici à un n'apprenant que les entités de plus haut niveau. En comparant les deux jeux d'annotations, nous pourrions voir quelles sont les erreurs spécifiques à la cascade et celles qui ne le sont pas. Nous avons également prévu de mesurer nos résultats avec la métrique plus récente appelée \textit{Entity Error Rate} (ETER) \cite{jannet2014eter}. Elle se base sur le SER mais prend mieux en compte la structuration.

Au moment de l'écriture de ces lignes, les résultats de la cascade \emph{bootstrap} n'ont pas encore pu être publiés, mais leur soumission à un colloque international est prévue.

En guise de perspective, nous pourrions également utiliser des méthodes de parsing comme celle décrite par \citet{dinarelli2012}. Une piste pour cela est celle des réseaux de neurones récursifs \citep{socher2011parsing} et plus particulièrement les \textit{Compositional Vector Grammars} (grammaires compositionelles de vecteurs) \citep{socher2013parsing}. Ce type de réseaux récursifs a déjà été utilisé pour parser des phrases selon une analyse syntaxique profonde, nous pourrions adapter cette méthode pour reconnaître les entités nommées structurées et comparer avec \citet{dinarelli2012} afin de voir l'apport d'une méthode neuronale dont nous avons montré le potentiel dans la section \ref{subsec:bootstrap-parsing}.

Comme nous l'avons dit précédemment, l'un des grands problèmes des annotations structurées est le manque de données. Une perspective serait de créer de nouvelles données annotées à l'aide des modèles existants en utilisant des techniques d'\emph{active learning} \citep{angluin1987learning,mackay1992information}, où le corpus est construit selon un processus de boucle de \textit{feedback}, l'algorithme proposant des exemples à un oracle qui les validera avant de les réinjecter dans le corpus d'apprentissage afin de générer petit à petit un corpus annoté. Cette perspective serait intéressante à plusieurs titre. Outre l'accélération de la création de nouvelles ressources, l'\emph{active learning} sur des annotations structurées pose également la question de la proposition des exemples à l'oracle et de leur validation. Valider des annotations structurées pose plusieurs questions. Doit-on valider de la racine aux feuilles ? Des feuilles aux racines ? Doit-on valider la structure complète en une seule passe ? Même la sélection des exemples apporte son lot de questions. Par exemple, doit-on choisir une entité par rapport à sa racine ou faut-il classifier des arbres entiers ? Ces questions ne seront malheureusement pas explorées pendant cette thèse, mais font partie des thématiques de recherches que je souhaiterais aborder dans le futur.

En supposant que les entités nommées ont été retrouvées, il devient possible d'établir des liens entre ces dernières, ces liens étant les \emph{relations} qu'elles ont les unes avec les autres. Nous pouvons voir cette extraction de relations entre entités nommées comme une forme de structuration également. Cette structuration est à l'échelle du document lorsque des mentions sont mises en relation. Cette extraction des relations entre entités nommées sera l'objectif du prochain chapitre.

\end{document}