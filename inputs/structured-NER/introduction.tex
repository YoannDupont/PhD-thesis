\documentclass[PhD-Yoann-Dupont.tex]{subfiles}
\begin{document}

Dans le chapitre précédent, nous avons vu la structuration des entités nommées d'un point de vue morphologique et syntagmatique. Dans ce chapitre, nous étudierons cette structuration d'un point de vue syntaxique. Les annotations dans ce type de modèle ont généralement une structure d'imbrications ou arborée, comme donné dans la figure \ref{fig:address-tree}. Une telle structure permet d'obtenir des annotations bien plus riches et informatives qu'une structure plate. Malgré une structure plus complexe, la tâche n'est pas plus difficile pour autant, elle devient même plus simple sur certains aspects. Comme nous l'avons vu pour les adresses, il est plus ardu de reconnaître l'intégralité d'une adresse que de reconnaître d'abord ses éléments constitutifs et d'en reconstruire la structure petit à petit, comme illustré dans la figure \ref{fig:address-tree}. Pour une des entités nommées de types plus "classique", cette structuration permet la reconnaissance d'entités qu'il serait particulièrement difficile à reconnaître autrement, comme illustré dans la figure \ref{fig:quaero-deepest}.

Il est à noter que la structuration des entités nommées est difficilement évitable : nous avons vu précédemment que l'utilisation de lexiques permettait d'améliorer la qualité des systèmes à base d'apprentissage en leur donnant des informations générales. Par exemple, nous savons humainement qu'une personne a généralement un prénom et un nom de famille et créons généralement un lexique pour chacun d'entre eux. Les tokens déclencheurs permettent également d'offrir une bonne généralisation. La présence d'un titre de civilité devant un nom propre est généralement considéré comme un contexte fort pour extraire une personne. Il en va de même pour les types de société (anonyme, à responsabilité limitée, etc...) ou d'organisation (organisation, fédération, union, etc...). Ces éléments sont dits \emph{constitutifs} d'une entité. La structuration des entités nommées peut alors se formuler comme la reconnaissance d'une entité et des différents éléments qui constituent cette dernière. Par exemple, «\ Maurice Arnoux\ » est une personne dont le prénom est «\ Maurice\ » et le nom est «\ Arnoux\ ». Dans le contexte plus large de l'adresse «\ 1 rue Maurice Arnoux\ », il s'agit également d'un nom de rue. Supposons maintenant que des personnes motivées du laboratoire Lattice décident d'organiser le «\ tournoi de tennis 1 rue Maurice Arnoux\ »\footnote{Le laboratoire Lattice étant situé au 1 rue Maurice Arnoux.}, l'adresse fera alors partie d'une entité de type évènement. Nous pouvons voir avec l'exemple précédent que la structure des entités est accumulative, se construisant petit à petit. Cet exemple d'une adresse peut s'étendre à d'autres entités, comme un complexe sportif, ou une société, pour lesquels il n'est pas rare qu'ils portent le nom d'une personne.

La notion d'entité nommée pose certaines questions quant à l'éventail des types intéressants à analyser. Différentes définitions aux étendues variables ont été proposées, recouvrant de 5 à 10 types \citep{grishman1996message,tjong2003introduction,sagot2012annotation} à environ 200 \citep{sekine2002extended}. Définir précisément ce qu'est une entité nommées et comment ces dernières sont construites pose encore plus de questions lorsque la structuration doit être prise en compte. Par exemple, la (non-)prise en compte du titre ou de la fonction pour les personnes changera son étendue et pourra ajouter le type "titre" à la liste de types reconnus. Les montants illustrent assez bien cette question du périmètre : doit-on annoter uniquement les montants monétaires (euros, dollars, etc...) ? Les montants mesurables selon une unité de mesure (heures, grammes, kilomètres/heure, honoraires) ? Ou l'ensemble des éléments quantifiés (quatre doctorants, huit cafés) ? La structuration des entités soulève également des questions qui lui sont propres. Dans le cadre classique, le nom d'un pays peut être utilisé pour désigner un lieu ou son gouvernement, chacun de ces cas étant annoté différemment. Lorsque les entités sont structurées, nous pouvons nous poser la question de comment annoter le gouvernement d'un pays. Doit-on d'abord annoter le pays avec une annotation de type organisation par dessus, puisque le gouvernement est indissociable de son pays ? Doit-on seulement annoter le gouvernement et pas le lieu, puisque nous parlons effectivement d'un gouvernement ? Bien que ce chapitre ne propose pas de réflexion approfondie sur le sujet de l'annotation structurée en soi, ces problématiques méritent d'être exposées car elles montrent les problèmes spécifiques à l'annotation en entités nommées structurées.

Nos contributions dans ce chapitre sont les suivantes. Premièrement, nous proposons un type de cascade d'étiqueteurs linéaires qui n'avait jusqu'à présent jamais été utilisée pour la reconnaissance d'entités nommées, généralisant les approches précédentes qui ne sont capables de reconnaître des entités de profondeur finie ou ne pouvant modéliser certaines particularités des entités nommées structurées. Nous apportons également comme contribution un comparatifs entre les CRF et les réseaux de neurones, où nous montrons un avantage clair des réseaux de neurones pour la tâche, ces derniers modélisant mieux les dépendances qui peuvent exister d'un niveau à l'autre. Les travaux que nous avons menés pour décrire ce chapitre nous ont permis de publier dans une conférence internationale, à savoir CICling 2017.

L'un des problèmes principaux concernant les corpus avec des annotations stucturées est leur disponibilité. En effet, ces corpus sont des ressources très rares, il n'en existe aucun pour l'anglais à notre connaissance. Dans ce chapitre, nous nous concentrerons sur le corpus Quaero, l'un des rares corpus à proposer des annotations en entités nommées structurées, qui sera l'objet principal de l'étude de ce chapitre.

\end{document}