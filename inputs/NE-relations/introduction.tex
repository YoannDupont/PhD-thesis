\documentclass[PhD-Yoann-Dupont.tex]{subfiles}
\begin{document}

Dans cette perspective, nous irons plus loin que l'extraction des entités nommées, en extrayant leurs relations. L'extraction de relations entre entités nommées est la suite logique de la reconnaissance des entités nommées. Il s'agit ici de structurer une connaissance à l'échelle, non plus des syntagmes, mais du document. Lorsque l'on extrait des relations entre entités nommées, la structuration que l'on peut extraire comme au chapitre \ref{chap:structured-NER} peut s'avérer particulièrement utile. Si l'on cherche par exemple à savoir si deux personnes sont apparentées, le fait qu'elle ait le même nom de famille est en général un bon indice. Un autre indice intéressant pour établir la relation entre deux entité est dans la présence de certains tokens. La présence d'un token comme "frère" ou "s\oe ur" est présent entre deux personnes est également à un indice\footnote{la présence seule du token est rarement suffisante pour décider avec certitude.} pour décider s'ils sont apparentées.

Dans cette perspective, nous donnerons une vision du thème "construction automatique d'une base de connaissance" du projet \textit{Multimedia Multilingual Integration} (IMM) de l'Institut de Recherche Technologique SystemX (IRT SystemX). Le projet étant assez conséquent et notre contribution assez spécifique, nous détaillerons donc également des contributions d'autres membres de l'IRT en plus des nôtres afin d'offrir le portrait minimal nécessaire pour la compréhension de la perspective. Dans le cadre de ce projet, nous avons aidé à produire une chaîne de traitements afin de construire automatiquement une \emph{base de connaissance} en participant au module d'extraction de relations entre mentions d'entités nommées. Ces travaux menés dans le cadre du projet IMM m'ont permis de participer au défi TAC-KBP 2016 \citep{rahman2017tac} ainsi que de contribuer à une démo à JEP-TALN 2016 \citep{mesnard2016}.

%La construction de la base de connaissance peut se voir (grossièrement) comme une passe de consolidation des différentes connaissances extraites dans des documents spécifiques d'un corpus. Cette base de connaissance peut se voir comme un graphe où les n\oe uds sont des entités et les arcs des relations qui lient deux entités entre elles.

Le projet IMM était pour moi une façon d'étudier l'une des applications directes des entités nommées, à savoir la recherche de relations entre entités, ce qui m'a permis de m'ouvrir à de nouvelles problématiques, de nouveaux défis ainsi qu'aux approches pour y répondre. J'étais à SystemX à hauteur de 20\% de mon temps (1 jour par semaine) où j'ai intégré de nouveaux modules dans le logiciel qui y était présent. Les travaux présentés dans cette perspective sont encore préliminaires.

Dans cette perspective, nous détaillerons d'abord la tâche à laquelle nous nous sommes attaquée, à savoir l'extraction de relations dans le cadre du défi TAC-KBP 2016. Nous détaillerons ensuite l'approche de \emph{supervision distante} avec laquelle nous avons constitué un corpus d'apprentissage pour répondre à la tâche. Puis, nous détaillerons l'outil que nous avons utilisé pour détecter les relations entre mentions d'entités nommées, à savoir MultiR, et de comment nous l'avons configuré pour répondre à la tâche. Nous détaillerons ensuite les résultats que nous avons obtenus et concluerons.

\end{document}