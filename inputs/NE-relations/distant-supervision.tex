\documentclass[PhD-Yoann-Dupont.tex]{subfiles}
\begin{document}

Les données annotées dans le TAL sont une ressource aussi rare que précieuse. Ces ressources tendent à être d'autant plus rares que la tâche est difficile. L'extraction de relations entre entités nommées est une tâche plutôt complexe en raison de la grande liberté quant aux relations qui peuvent lier différentes entités nommées. Il y a en effet très peu de relations typiques reliant les entités nommées. À la difficulté normale de constitution d'un corpus annoté s'ajoute donc celle d'une tâche très générique, dont l'étendue se révèle être très variable. Démarrer de zéro l'annotation d'un corpus en relations entre entités nommées est particulièrement long et fastidieux, les entités nommées devant auparavant être annotées. Il est possible de reprendre un corpus annoté en entités nommées, mais cela limiterait grandement les possibilités des relations qu'il est possible d'extraire, toutes les entités nommées d'intérêt n'étant pas nécessairement annotées. Par exemple, si nous cherchons les lieux de naissance des différentes personnes dans un corpus, les lieux d'intérêt peuvent être des hôpitaux, donc des bâtiments, qui sont rarement annotés.

Contrairement aux machines, les humains ont un lot de connaissances qu'ils peuvent utiliser (ou acquérir) pour répondre à une tâche. Si la tâche s'apparente à de la classification ou de l'identification, les humains sont notamment capables de donner des exemples de représentants de chaque classe. Par exemple, un être humain peut citer des personnes, des lieux, etc. Le même principe s'applique aux relations entre entités : un être humain est capable de donner des personnes nées à un endroit particulier, des personnes mariées, etc. Ces connaissances peuvent alors être utilisées afin de trouver des représentants potentiels dans un corpus. Par exemple, si l'on sait que "France" est un pays, alors il est possible de considérer l'ensemble des occurrences de "France" comme un pays potentiel. De la même façon, il peut également citer des personnes nées en France, les personnes mariées, la nationalité d'une entreprise, etc. Cet ensemble de connaissances peut être utilisée pour récupérer des ensembles de phrases d'exemples qu'il est alors possible d'utiliser comme un corpus d'apprentissage. Le principe de générer un corpus annoté à partir d'un ensemble de connaissances est appelé la \emph{supervision distante}.

La supervision distante utilise des \emph{bases de connaissances}, qui sont des bases de données servant à emagasiner des données plus ou moins structurées représentant des faits, ici des relations entre entités notées $r(e_{1},e{2})$. Par exemple, une base de connaissances peut contenir le fait $fondateur(Richard\ Stallman,\ Free\ Software\ Foundation)$, qui s'interprète comme «\ Richard Stallman est le fondateur de Free Software Foundation\ ». Considérant ce fait, nous pouvons récupérer l'ensemble des phrases contenant «\ Richard Stallman\ » et «\ Free Software Foundation\ » et supposer qu'elles expriment la relation "fondateur de" entre les mentions des deux entités. Si nous explorons par exemple Wikipedia, nous pouvons trouver des phrases comme «\ En 1985, Richard Stallman crée la Free Software Foundation (FSF)\ ». En utilisant de larges bases de connaissances et de larges quantités de textes, il est ainsi possible de créer un corpus annoté à moindre effort. L'un des problèmes de la supervision distante est l'absence de garantie par rapport à la qualité des phrases proposées par la méthode.


\end{document}