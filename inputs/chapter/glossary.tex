\documentclass[PhD-Yoann-Dupont.tex]{subfiles}
\begin{document}

\section*{A}
\textbf{Affixe caractéristique (d'un token)} : affixe d'un token aidant à sa désambiguisation dans le cadre d'une tâche d'étiquetage.

\textbf{Analyse syntaxique} : processus par lequel la structure d'un texte est retrouvée.

\section*{B}
\textbf{Bootstrapped} : voir \textbf{cascade bootstrapped}

\section*{C}
\textbf{Cascade bootstrapped} : cascade d'étiqueteurs appliquée de manière récursive.

\textbf{Cascade kickstarted} : cascade d'étiqueteurs se divisant en deux passes. La première passe, non-récursive, fournit des annotation qui serviront de contexte à la seconde passe, quant à elle récursive.

\section*{D}
\textbf{Descripteur} : information relative à un token, générable de manière algorithmique.

\section*{G}
\textbf{Grammaire Hors Contexte Probabiliste} : méthode d'apprentissage automatique permettant de parser une séquence d'éléments selon une structure arborée.

\section*{K}
\textbf{Kickstarted} : voir \textbf{cascade kickstarted}

%\section*{M}
%\textbf{Morphème} : unité linguistique atomique ayant une forme et un sens.

\section*{P}
\textbf{PCFG} : Probabistic Context-Free Grammar (voir \textbf{Grammaire Hors Contexte Probabiliste})

\section*{R}
\textbf{Répertoire (de lexiques)} : ensemble de lexiques classés et triés, utilisés pour générer des descripteurs. Un répertoire peut également fournir des descripteurs ambigus si un même token appartient à deux lexiques différents.

\section*{T}
\textbf{Token} : unité de segmentation atomique dans le traitement des séquences.

\textbf{Token déclencheur} : token servant de contexte fort à une entité nommée.

\end{document}