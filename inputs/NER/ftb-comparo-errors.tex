\documentclass[PhD-Yoann-Dupont.tex]{subfiles}
\begin{document}

Les systèmes à base de règles ont des extractions plutôt similaires, et partagent la même erreur principale sur le FTB : un fort silence sur les organisations, dont le rappel atteint à peine les 50\%. La seconde erreur que les systèmes à base de règles partagent sur ce corpus est relatif à la structuration des organisations dont le nom contient un pays. C'est par exemple le cas du «\ Crédit Foncier de France\ », où les systèmes à base de règle annotent «\ Crédit Foncier\ » en tant qu'organisation et «\ France\ » en tant que pays. Le reste des erreurs est principalement du à des différences de périmètre en termes d'entités nommées. Par exemple, certains pays comme l'URSS n'existent plus à l'heure actuelle, mais cela était encore le cas à l'époque de certains articles, où ils sont encore annotés, ce qui cause des erreurs de silence pour les systèmes à base de règle. les chaînes télévisées ne sont pas nécessairement annotées, tout comme les pays n'existant plus comme l'URSS. Ces systèmes reconnaissent également les indices boursiers, qui ne sont pas annotés dans le FTB, ce qui cause une augmentation de leur bruit.

Le CRF a, quant à lui, tendance à commettre des erreurs sur les entités inconnues, ainsi que sur les ambiguités entre lieux et organisations, comme les pays et les villes. Une source de bruits pour le CRF est sur les abréviations tout en majuscules, confondues avec des formes courtes d'entité. Une source d'erreur de frontière est sur les variantes d'entités qui comprennent, ou non, un déterminant, comme c'est le cas pour «\ Le Monde\ », qui est annoté comme une entité, alors que «\ le Monde\ » n'aura que «\ Monde\ » d'annoté. Certains tokens déclencheurs peuvent parfois être source d'erreur, comme «\ saint\ », utilisé pour les personnes, cause l'extraction «\ Saint Louis\ » comme personne à la place de lieu.

Bi-LSTM-CRF a également des erreurs qui lui sont propres. En ce qui concerne les personnes, les erreurs ont tendance à être contextuelles. Par exemple, Bi-LSTM-CRF n'a pas su distinguer «\ Maurice\ » de  «\ l'Île Maurice\ », ce qui a engendré par la suite des erreurs contextuelles comme par exemple «\ l'Île Maurice et Tamatave\ », où Bi-LSTM-CRF a bien identifié que Tamatave était une entité, mais où l'erreur de typage s'est propagée, «\ Tamatave\ » étant alors identifié comme une personne également. Bi-LSTM-CRF a également eu du mal à annoter certains tokens étrangers, ce qui a donné lieu à des erreurs où il y avait alors une ambiguité entre une personne et une organisation. Dans l'exemple «\ [...] réservées aux médecins comme Canal Santé ou aux informaticiens comme Computer Channel\ », les entités «\ Canal Santé\ » et «\ Computer Channel\ » ont été annotées comme étant des personnes et non des organisations. Cette utilisation du contexte a également permis la bonne désambiguisation de certaines entités : par exemple «\ le groupe Berlusconi\ », où «\ Berlusconi\ » est annoté dans le FTB comme étant une personne, a été annoté ici comme étant une organisation. D'un point de vue général, Bi-LSTM-CRF est, par rapport au CRF, un système plus couvrant mais également plus bruyant.

\end{document}