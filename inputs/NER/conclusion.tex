\documentclass[PhD-Yoann-Dupont.tex]{subfiles}
\begin{document}

Dans cette section, nous avons vu les différentes méthodes de base permettant d'effectuer la reconnaissance d'entités nommées dans sa définition la plus simple. Les méthodes présentées ne sont pas nécessairement adaptées à des tâches où une forte analyse morphologique et/ou syntaxique est requise, comme pour les corpus Genia ou CHEMDNER. Ces méthodes ne sont également pas nécessairement adaptées à la reconnaissance d'entités nommées dont la structuration n'est pas linéaire, comme pour le Quaero. Dans les prochains chapitres, nous aborderons principalement la thématique de la structuration dans les entités nommées, cette dernière se déclinant en deux parties principales : une structuration morphologique et une syntaxique. Les enjeux de la première sont d'analyser la structure de la forme fléchie d'une entité, à ce titre une entité chimique complexe se décrit comme l'ensemble des éléments chimiques simples qui la constituent, ainsi que la recherche de contextes déclencheurs d'une entité. Ceux de la deuxième consistent à retrouver la structure interne d'une entité, c'est-à-dire retrouver les différents éléments qui la constituent, comme une personne est constituée d'un prénom et d'un nom. Dans le chapitre suivant, nous étudierons la morphologie des entités nommées.

\end{document}