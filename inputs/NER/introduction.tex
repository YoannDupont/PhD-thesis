\documentclass[PhD-Yoann-Dupont.tex]{subfiles}
\begin{document}

Si l'on revient sur les diverses définitions d'entitées nommées données dans la section\ \ref{sec:NE-introduction}, il est aisé de voir qu'il s'agit d'une notion très liée à la syntaxe\footnote{la REN souvent vue comme une tâche de syntaxe, non de sémantique}. Dans cette partie, nous présenterons un éventail des méthodes utilisées afin de répondre à la tâche de la reconnaissance d'entités nommées. Il existe deux types de méthodes afin de répondre à cette tâche. La première consiste à écrire des programmes représentant le raisonnement d'un être humain afin d'identifier une entité, ces systèmes sont appelés des systèmes à base de règles, ces derniers modélisant généralement des conditions dans lesquelles une décision peut être prise de façon sûre. Une autre méthode consiste à appliquer des algorithmes auxquels nous allons donner des exemples afin qu'ils infèrent eux-mêmes des éléments de décision afin de pouvoir au mieux reproduire les exemples à leur disposition, cette approche s'appelant l'\emph{apprentissage automatique}.

Nous commencerons par détailler les différentes métriques de qualité qu'il est possible d'utiliser pour évaluer les système effectuant la REN. Nous parlerons ensuite de différents systèmes à base de règles. Nous continuerons ensuite en détaillant deux approches par apprentissage, à savoir les CRF et les réseaux de neurones. Nous terminerons ensuite sur un comparatif des différentes méthodes sur le French Treebank.
%Avant de détailler les différentes approches utilisées dans le TAL pour effectuer de la REN, nous commencerons par détailler les différentes mesures de qualité qu'il est possible d'utiliser pour cette tâche, qui sont les mêmes quelle que soit la méthode utilisée.

\end{document}